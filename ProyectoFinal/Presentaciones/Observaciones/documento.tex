\documentstyle[11pt]{report}
%\setcounter{page}{6}

\oddsidemargin 0pt \evensidemargin 0pt
\topmargin=1.25in
\headheight 10pt \headsep 10pt \footheight 10pt \footskip 24pt
\textheight 10in \textwidth 6.5in \columnsep 10pt \columnseprule 0pt

\font\namefont=cmr10 scaled\magstep2
\voffset=-.75in
\parskip=.075in
\parindent=0in

\thispagestyle{empty}
\begin{document}
\bigskip



\bigskip
\large \centerline {\namefont \bf DOCUMENTO DE OBSERVACIONES, CONCLUSIONES} 
\large \centerline {\namefont \bf Y EXPERIENCIAS DEL DESARROLLO DE NUESTRA}
\large \centerline {\namefont \bf APLICACION ANDROID}
\bigskip

\centerline{\namefont  \small Cuadrado Daniel, Mite Juan, Torres Criollo Daniel, V\'elez Jos\'e}
\bigskip

\vspace{.1 in}
\hrule
\makebox[3.5in][l]


{\leftskip=.6in  \parindent=-.3in  \parskip=.05in

\bigskip

\centerline {\namefont \LARGE \bf SEARCH AND SAVE}
\bigskip
\bigskip
\bigskip

{\bf DESCRIPCION DE LA APLICACION}
\bigskip
\\La aplicaci\'on se ejecuta en su mayor\'ia en segundo plano, cuando el usuario ingresa tema de consulta y lo guarda, se ejecuta el navegador y encuentra la primera incidencia de lo buscado y guarda la p\'agina, la almacena en la micro SD, elimina el tema de consulta de la lista de notas y as\'i vuelve a realizar este proceso hasta que la lista de b\'usqueda quede vac\'ia.
\bigskip
\bigskip

{\bf FUNCIONALIDADES DE LA APLICACION}
\bigskip
\\La aplicaci\'on SEARCH-SAVE resolver\'a el problema de buscar t\'erminos desconocidos por el usuario. La aplicaci\'on recibe el tema de consulta, y en segundo plano, \'esta se encarga de buscar la informaci\'on disponible en l\'nea y el usuario la puede revisar cuando quiera.   
\bigskip
\bigskip

{\bf OBSERVACIONES}
\bigskip
\\Desarrollar en Android por primera vez requiere de mucha paciencia, tener conocimiento de Java y programaci\'on en HTML por esto  de las etiquetas, tener mucho cuidado con los permisos porque puedes pasar todo un d\'ia buscando el error sem\'antico  que no existe en un bloque de c\'odigo para al final darte cuenta que no habí\'is solicitado un permiso, tambi\'en revisar el archivo MANIFEST es aqu\'i donde se guarda la estructura esqueleto de la app y donde se deben definir las funcionalidades que se requiere implementar.   
\bigskip
\bigskip

{\bf CONCLUSIONES}
\bigskip
\\Aunque muchos expertos aseguran que en 30 a\~nos Android ser\'a el mejor el sistema operativo del planeta, a\'un le queda un arduo camino por recorrer para alcanzar la excelencia, sin embargo programar sobre \'el es relativamente complicado y ma\~noso, decimos ma\~noso porque hay veces que corre sin ning\'un problema y hay veces que no, hay veces que emula sin problemas y hay veces que “un hilo se rompe” , hay veces que corre en otra m\'aquina y hay veces que no, en fin hemos requerido de paciencia sobrehumana y dedicaci\'on para llevar a cabo esta aplicaci\'on, disfr\'utenla.
\bigskip
\bigskip

\newpage
{\bf EXPERIENCIAS}
\bigskip
\\No muy buenas, aunque al principio todos nos emocionamos puesto que ninguno de nosotros antes hab\'ia programado en ANDROID,  luego comenzamos a tener los primeros inconvenientes con el IDE y dem\'as cuestiones de instalaci\'on al parecer la plataforma necesita muchos recursos, es pesada y de vez en cuando muestra errores en donde existen, Lo mejor de trabajar sobre Android fue que en internet hay muchos soportes, foros, etc. Aunque los mejores estaban en ingles hab\'ia la ayuda suficiente para solucionar m\'as de un problema. 
\\Nos vimos entusiasmados al realizar el prototipo no funcional de la aplicaci\'on puesto que nos result\'o muy f\'acil, levantar frames y conectarlas entre si no era cosa de otro mundo, la parte complicada y los dolores de cabeza nos llegar\'ian despu\'es, cuando empezamos a implementar las funcionalidades, como nos supo advertir profesor, no era trivial la funcionalidad de nuestra aplicaci\'on, es aqu\'i donde empezaron los dolores de cabeza y las malas noches con la b\'usqueda de ejemplos de cómo guardar en pdf, como descargar el HTML de una p\'agina.  
\\Como a dos integrantes de nuestro equipo no ten\'ian internet en sus casas no pudimos darle el uso que usted hubiera querido a la herramienta de colaboraci\'on github.

\end{document}
